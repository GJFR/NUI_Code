%%%% ijcai11.tex

\typeout{IJCAI-11 Instructions for Authors}

% These are the instructions for authors for IJCAI-11.
% They are the same as the ones for IJCAI-07 with superficical wording
%   changes only.

\documentclass{article}
% The file ijcai11.sty is the style file for IJCAI-11 (same as ijcai07.sty).
\usepackage{ijcai11}
\usepackage{mathtools}
% The language we want and appropriate hyphenation.
\usepackage[dutch]{babel}

% Hyphenation rules.
\hyphenation{biopotentiaal-schommelingen}

% Use the postscript times font!
\usepackage{times}

% the following package is optional:
%\usepackage{latexsym} 


\title{TODO Titel}
\author{Kevin Boets \\ Katholieke Universiteit Leuven\\ Leuven, Belgium \\ kevin.boets@student.kuleuven.be
\And
Gertjan Franken \\ Katholieke Universiteit Leuven\\ Leuven, Belgium \\ gertjan.franken@student.kuleuven.be}

\begin{document}

\maketitle

\begin{abstract}
 TODO
  The {\it IJCAI--11 Proceedings} will be printed from electronic
  manuscripts submitted by the authors. The electronic manuscript will
  also be included in the online version of the proceedings. This paper
  provides the style instructions.
\end{abstract}

\section{Introductie}

TODO 

\section{Preprocessing}

De data die we krijgen doorgestuurd, afkomstig van de sensoren, is zeer ruw. Om hier kijkrichtingen uit af te kunnen leiden, gaan we de data eerst beter leesbaar maken. Dit is de preprocessing-stap. Deze stap voeren we op elke dataset uit alvorens we deze gaan analyseren.

Als eerste gaan we de ruis en knipperingen zo goed mogelijk proberen te verminderen. Dit gebeurt door middel van een low-pass filter. Deze filter verzacht alle signalen waarvan de frequentie hoger is dan onze cutoff frequentie. [TODO berekenen wat onze cutoff frequentie is].

Daarna focussen we ons op de biopotentiaalschommelingen. Deze schommelingen kunnen we niet op voorhand voorspellen en verschillen van persoon tot persoon. Als we deze schommelingen kunnen verminderen, kunnen we meer steunen op absolute waardes van het signaal. In deze tweede stap van de preprocessing gebruiken we opnieuw een low-pass filter. Deze keer gebruiken we een lage cutoff frequentie [TODO weer cutoff berekenen]. Na deze filtering van de data, blijven enkel de signalen met een lage frequentie over. Het resultaat is ongeveer de biopotentiaal van de gebruiker die doorheen de tijd fluctueert. We kunnen nu de biopotentiaalschommelingen uit onze originele data verwijderen door de gevonden biopotentiaal hiervan af te trekken. De schommelingen verdwijnen niet, maar worden wel sterk verminderd.

Na toepassing van de vorige filteringen, discretiseren we de data. Hierdoor moeten we later minder berekeningen doen, wat de uitvoeringstijd van het programma ten goede komt. Hiervoor gebruiken we de SAX (Symbolic Aggregate approXimation) \cite{sax}. [TODO rest SAX]

[TODO slot?]

\section{Gebruikte methoden}

In ons onderzoek hebben we gebruik gemaakt van twee verschillende methoden om kijkrichtingen te herkennen. Deze methoden steunen respectievelijk op thresholds en patronen. Beide methode beginnen met een calibratie-fase, gevolgd door de herkennings-fase. Tijdens de calibratie-fase krijgen we onze eerste data en proberen daaruit informatie te halen. In de herkennings-fase gaan we deze gevonden informatie gebruiken om kijkrichtingen te detecteren.

\subsection{Thresholds}

Deze methode steunt op de absolute waarden van de data. Er wordt een bovengrens gedefinieerd voor elke kijkrichting in de calibratie-fase. Als deze overschreden wordt in de herkennings-fase, gaan we uit van die bepaalde kijkrichting.

In de calibratie-fase wordt aan de gebruiker gevraagd om een bepaalde sequentie van kijkrichtingen uit te voeren. Uit deze dataset halen we de kleinste en grootste waarde. Deze twee waarden horen elk bij een andere kijkrichting. Per kijkrichting hebben we ook een constante $\alpha$ waarvoor geldt: $0 < \alpha \leq 1$. We vermenigvuldigen $\alpha$ met dit minimum en maximum en beschouwen de uitkomten als de uiteindelijke bovengrens van de respectievelijke kijkrichtingen. In ons programma hebben we $\alpha$ voor beide kijkrichtingen de waarde $0.5$ gegeven, maar deze kan veranderd worden naar behoefte. Zo kunnen we de methode bijvoorbeeld strenger maken door $\alpha$ te verhogen.

In de herkennings-fase gaan we met behulp van deze bovengrenzen kijkrichtingen proberen te detecteren. We nemen telkens het gemiddelde van tien nieuwe punten. Dit gemiddelde gaan we dan vergelijken met alle bovengrenzen. Wanneer een bovengrens overschreden wordt, toont dit aan dat er mogelijk een kijkrichting bezig is. We zouden dit kunnen signaleren als een kijkrichting, maar er is nog onzekerheid. Om voor meer zekerheid te zorgen, wachten we tot de grens meerdere keren is overschreden. Dit aantal is een afweging tussen hoge zekerheid of hoge detectiekans. Bij een klein aantal is de kans op detectie hoog, maar we zijn minder zeker ofdat het geen knippering is. Een groot aantal zal anderzijds veel zekerheid verschaffen, waarbij er wel een grotere kans is dat korte kijkrichtingen niet worden gedetecteerd. We moeten dus een goede balans proberen te vinden.

Bij veelvuldige herhaling van dezelfde kijkrichting, wordt een nieuw probleem zichtbaar.

\subsection{Patronen}
TODO

\section{Implementatie}
TODO

\section{Resultaten}
TODO

\section{Conclusie}
TODO

\section*{Acknowledgments}

The preparation of these instructions and the \LaTeX{} and Bib\TeX{}
files that implement them was supported by Schlumberger Palo Alto
Research, AT\&T Bell Laboratories, and Morgan Kaufmann Publishers.
Preparation of the Microsoft Word file was supported by IJCAI.  An
early version of this document was created by Shirley Jowell and Peter
F. Patel-Schneider.  It was subsequently modified by Jennifer
Ballentine and Thomas Dean, Bernhard Nebel, and Daniel Pagenstecher.
These instructions are the same as the ones for IJCAI--05, prepared by
Kurt Steinkraus, Massachusetts Institute of Technology, Computer
Science and Artificial Intelligence Lab.

\appendix

\section{\LaTeX{} and Word Style Files}\label{stylefiles}

The \LaTeX{} and Word style files are available on the IJCAI--11
website, {\tt http://www.ijcai-11.org/}.
These style files implement the formatting instructions in this
document.

The \LaTeX{} files are {\tt ijcai11.sty} and {\tt ijcai11.tex}, and
the Bib\TeX{} files are {\tt named.bst} and {\tt ijcai11.bib}. The
\LaTeX{} style file is for version 2e of \LaTeX{}, and the Bib\TeX{}
style file is for version 0.99c of Bib\TeX{} ({\em not} version
0.98i). The {\tt ijcai11.sty} file is the same as the {\tt
ijcai07.sty} file used for IJCAI--07.

The Microsoft Word style file consists of a single file, {\tt
ijcai11.doc}. This template is the same as the one used for
IJCAI--07.

These Microsoft Word and \LaTeX{} files contain the source of the
present document and may serve as a formatting sample.  

Further information on using these styles for the preparation of
papers for IJCAI--11 can be obtained by contacting {\tt
pcchair11@ijcai.org}.

%% The file named.bst is a bibliography style file for BibTeX 0.99c
\bibliographystyle{named}
\bibliography{paper}

\end{document}

